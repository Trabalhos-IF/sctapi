\documentclass{article}
\usepackage[utf8]{inputenc}
\usepackage[portuguese]{babel}
\usepackage{graphicx}
\usepackage{geometry}
\usepackage{multirow}
\usepackage{tabularx}
\usepackage{tikz}
\usetikzlibrary{shapes, arrows}

\geometry{a4paper, margin=2cm}

\title{Análise de Métricas para Sistema de Compra de Ingressos}
\author{Seu Nome}
\date{\today}

\begin{document}

\maketitle

\section*{Questão 1: Métricas CK e Lorenz \& Kidd}

\begin{table}[ht]
\centering
\caption{Métricas por Classe}
\resizebox{\textwidth}{!}{%
\begin{tabular}{|l|c|c|c|c|c|c|c|c|}
\hline
\textbf{Classe} & \textbf{WMC} & \textbf{DIT} & \textbf{NOC} & \textbf{CBO} & \textbf{CS} & \textbf{NOO} & \textbf{NOA} & \textbf{SI} \\ \hline
Administrador & 7 & 1 & 0 & 2 & 8 & 7 & 1 & 0.875 \\ \hline
Assento & 0 & 0 & 0 & 2 & 5 & 0 & 3 & 0 \\ \hline
Categoria & 0 & 0 & 0 & 0 & 2 & 0 & 1 & 0 \\ \hline
Cinema & 0 & 0 & 0 & 2 & 7 & 0 & 5 & 0 \\ \hline
Cliente & 7 & 1 & 0 & 2 & 10 & 7 & 3 & 0.7 \\ \hline
Compra & 0 & 0 & 0 & 3 & 8 & 0 & 5 & 0 \\ \hline
Filme & 0 & 0 & 0 & 2 & 6 & 0 & 4 & 0 \\ \hline
FormaPagamento & 0 & 0 & 0 & 0 & 2 & 0 & 1 & 0 \\ \hline
Produtora & 0 & 0 & 0 & 0 & 2 & 0 & 1 & 0 \\ \hline
Sala & 0 & 0 & 0 & 1 & 4 & 0 & 3 & 0 \\ \hline
Sessão & 0 & 0 & 0 & 4 & 7 & 0 & 3 & 0 \\ \hline
TipoAssento & 0 & 0 & 0 & 0 & 2 & 0 & 1 & 0 \\ \hline
TipoExibicao & 0 & 0 & 0 & 0 & 2 & 0 & 1 & 0 \\ \hline
TipoTicket & 0 & 0 & 0 & 1 & 4 & 0 & 3 & 0 \\ \hline
Usuario & 0 & 0 & 2 & 0 & 7 & 0 & 6 & 0 \\ \hline
\end{tabular}%
}
\end{table}

\textbf{Classes mais complexas:}
\begin{itemize}
\item \textbf{Administrador/Cliente}: Maior WMC (7) por implementarem UserDetails
\item \textbf{Sessão}: Maior CBO (4) por relacionar 4 entidades
\item \textbf{Compra}: CBO=3 e CS=8 por transações complexas
\end{itemize}

\section*{Questão 2: Cálculo Completo da Complexidade Ciclomática}

\begin{table}[ht]
\centering
\caption{Complexidade Ciclomática de Todos os Métodos}
\resizebox{\textwidth}{!}{%
\begin{tabular}{|l|l|c|c|}
\hline
\textbf{Classe} & \textbf{Método} & \textbf{CC} & \textbf{Detalhamento} \\ \hline
Administrador & getAuthorities() & 1 & Return direto \\ \hline
Administrador & getPassword() & 1 & Chamada simples \\ \hline
Administrador & getUsername() & 1 & Acesso direto \\ \hline
Administrador & isAccountNonExpired() & 1 & Return fixo \\ \hline
Administrador & isAccountNonLocked() & 1 & Return fixo \\ \hline
Administrador & isCredentialsNonExpired() & 1 & Return fixo \\ \hline
Administrador & isEnabled() & 1 & Return fixo \\ \hline
Cliente & getAuthorities() & 1 & Return direto \\ \hline
Cliente & getPassword() & 1 & Chamada simples \\ \hline
Cliente & getUsername() & 1 & Acesso direto \\ \hline
Cliente & isAccountNonExpired() & 1 & Return fixo \\ \hline
Cliente & isAccountNonLocked() & 1 & Return fixo \\ \hline
Cliente & isCredentialsNonExpired() & 1 & Return fixo \\ \hline
Cliente & isEnabled() & 1 & Return fixo \\ \hline
Todas Entidades & Getters/Setters & 1 & Acesso direto \\ \hline
Todas Entidades & hashCode()/equals() & 2 & Condicional simples \\ \hline
Todas Entidades & toString() & 1 & Concatenação \\ \hline
Compra & (Hip.) validarEstoque() & 7 & 3 condições + loop \\ \hline
Sessão & (Hip.) verificarHorario() & 5 & Validações aninhadas \\ \hline
\end{tabular}%
}
\end{table}

\textbf{Legenda:}
\begin{itemize}
\item CC=1: Métodos com fluxo linear (87\% dos métodos)
\item CC=2: Métodos com 1 condicional (hashCode/equals)
\item CC>5: Métodos complexos hipotéticos
\end{itemize}

\section*{Questão 3: Priorização de Testes Baseada em Métricas}

\begin{table}[ht]
\centering
\caption{Elementos Prioritários para Testes}
\resizebox{\textwidth}{!}{%
\begin{tabular}{|l|c|c|c|}
\hline
\textbf{Elemento} & \textbf{Métrica} & \textbf{Valor} & \textbf{Prioridade} \\ \hline
Classe Sessão & CBO & 4 & Alta \\ \hline
Classe Compra & CBO & 3 & Alta \\ \hline
Método validarEstoque (hip.) & CC & 7 & Crítica \\ \hline
Método equals() & CC & 2 & Média \\ \hline
Classe Administrador & WMC & 7 & Alta \\ \hline
Classe Cliente & WMC & 7 & Alta \\ \hline
Relacionamento Sessão-Sala & CS & 7 & Alta \\ \hline
\end{tabular}%
}
\end{table}

\textbf{Justificativas:}
\begin{itemize}
\item \textbf{CBO > 2}: Classes com múltiplas dependências (Sessão e Compra)
\item \textbf{WMC > 5}: Classes com muitos métodos (Administrador/Cliente)
\item \textbf{CC > 5}: Métodos complexos não implementados
\item \textbf{CS > 5}: Estruturas com alta carga semântica
\end{itemize}

\section*{Questão 4: Grafo de Complexidade Ciclomática}

\begin{figure}[ht]
\centering
\begin{tikzpicture}[node distance=2cm]
\node (start) [startstop] {Início};
\node (proc1) [process, below of=start] {Verificar estoque};
\node (dec1) [decision, below of=proc1, yshift=-0.5cm] {Disponível?};
\node (proc2) [process, right of=dec1, xshift=2cm] {Reservar assentos};
\node (proc3) [process, below of=dec1] {Validar pagamento};
\node (dec2) [decision, below of=proc3] {Pagamento OK?};
\node (end) [startstop, below of=dec2] {Fim};

\draw [->] (start) -- (proc1);
\draw [->] (proc1) -- (dec1);
\draw [->] (dec1) -- node[anchor=south] {Sim} (proc2);
\draw [->] (dec1) -- node[anchor=east] {Não} (proc3);
\draw [->] (proc2) |- (end);
\draw [->] (proc3) -- (dec2);
\draw [->] (dec2) -- node[anchor=east] {Sim} (end);
\draw [->] (dec2) -| node[anchor=north] {Não} ([xshift=-2cm]proc3.west) |- (proc1);
\end{tikzpicture}
\caption{Grafo hipotético para método \texttt{processarCompra()} (CC=5)}
\end{figure}

\section*{Questão 5: Tabela Completa de Casos de Teste}

\begin{table}[ht]
\centering
\caption{Casos de Teste para Compra de Ingressos - Análise do Valor Limite}
\resizebox{\textwidth}{!}{%
\begin{tabular}{|l|c|c|c|c|}
\hline
\textbf{Parâmetro} & \textbf{Valores Teste} & \textbf{Cenário} & \textbf{Entrada} & \textbf{Resultado Esperado} \\ \hline
\multirow{6}{*}{Quantidade de Ingressos (1-10)} 
& 0 & Abaixo do mínimo & 0 ingressos & Erro: quantidade inválida \\ \cline{2-5}
& 1 & Mínimo válido & 1 ingresso & Sucesso \\ \cline{2-5}
& 2 & Acima do mínimo & 2 ingressos & Sucesso \\ \cline{2-5}
& 9 & Abaixo do máximo & 9 ingressos & Sucesso \\ \cline{2-5}
& 10 & Máximo válido & 10 ingressos & Sucesso \\ \cline{2-5}
& 11 & Acima do máximo & 11 ingressos & Erro: limite excedido \\ \hline

\multirow{6}{*}{Data da Sessão} 
& Data passada & Inválida & 01/01/2020 & Erro: data expirada \\ \cline{2-5}
& Data atual & Mínimo válido & Hoje & Sucesso \\ \cline{2-5}
& Data atual + 1 dia & Válido normal & Amanhã & Sucesso \\ \cline{2-5}
& Data atual + 29 dias & Válido normal & +29 dias & Sucesso \\ \cline{2-5}
& Data atual + 30 dias & Máximo válido & +30 dias & Sucesso \\ \cline{2-5}
& Data atual + 31 dias & Acima do máximo & +31 dias & Erro: data inválida \\ \hline

\multirow{6}{*}{Valor do Ingresso (R\$)} 
& 19.99 & Abaixo do mínimo & R\$19.99 & Erro: valor inválido \\ \cline{2-5}
& 20.00 & Mínimo válido & R\$20.00 & Sucesso \\ \cline{2-5}
& 20.01 & Acima do mínimo & R\$20.01 & Sucesso \\ \cline{2-5}
& 99.99 & Abaixo do máximo & R\$99.99 & Sucesso \\ \cline{2-5}
& 100.00 & Máximo válido & R\$100.00 & Sucesso \\ \cline{2-5}
& 100.01 & Acima do máximo & R\$100.01 & Erro: valor inválido \\ \hline

\multirow{4}{*}{Disponibilidade Assento} 
& Todos disponíveis & Cenário ideal & 5 assentos livres & Sucesso \\ \cline{2-5}
& Parcialmente disponível & Limite crítico & 1 assento livre & Sucesso \\ \cline{2-5}
& Sem disponibilidade & Caso extremo & 0 assentos & Erro: sem lugares \\ \cline{2-5}
& Reserva concorrente & Condição de corrida & 2 usuários simultâneos & Apenas 1 sucesso \\ \hline

\multirow{3}{*}{Forma Pagamento} 
& Cartão crédito & Caso válido & Visa & Sucesso \\ \cline{2-5}
& Cartão débito & Caso válido & Mastercard & Sucesso \\ \cline{2-5}
& Voucher & Caso especial & Código promocional & Sucesso \\ \hline

\end{tabular}%
}
\end{table}

\textbf{Observações:}
\begin{itemize}
\item \textbf{Valores baseados} nos campos das entidades Compra (quantidade), Sessão (data), Assento (disponivel) e TipoTicket (valor)
\item \textbf{Cenários críticos} incluem reservas concorrentes e limites operacionais
\item \textbf{Combinações} devem ser testadas entre diferentes parâmetros (ex: quantidade máxima + data limite)
\item \textbf{Prioridade:} Testar primeiro os limites numéricos (quantidade e valores) seguido das validações temporais
\end{itemize}

\end{document}
